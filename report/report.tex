\documentclass[acmtog]{acmart}
\usepackage{graphicx}
\usepackage{subfigure}
\usepackage{natbib}
\usepackage{listings}
\usepackage{bm}
\usepackage{amsmath}

\definecolor{blve}{rgb}{0.3372549 , 0.61176471, 0.83921569}
\definecolor{gr33n}{rgb}{0.29019608, 0.7372549, 0.64705882}
\makeatletter
\lst@InstallKeywords k{class}{classstyle}\slshape{classstyle}{}ld
\makeatother
\lstset{language=C++,
	basicstyle=\ttfamily,
	keywordstyle=\color{blve}\ttfamily,
	stringstyle=\color{red}\ttfamily,
	commentstyle=\color{magenta}\ttfamily,
	morecomment=[l][\color{magenta}]{\#},
	classstyle = \bfseries\color{gr33n}, 
	tabsize=2
}
\lstset{basicstyle=\ttfamily}

% Title portion
\title{Assignment 1:\\ {Exploring OpenGL Programming}} 


\author{Name: Lin Jiale \quad \\ student number: 2022533043 \
\\email: \quad \texttt{linjl2022@shanghaitech.edu.cn}}

% Document starts
\begin{document}
\maketitle

\vspace*{2 ex}

\section{Introduction}

\subsection{Ray-Triangle Intersection}

Complete the implementation of the ray-triangle intersection function \texttt{TriangleIntersect} in \texttt{src/accel.cpp}.

\subsection{Ray-AABB Intersection}

Complete the AABB (Axis-Aligned Bounding Box) intersection in the function \texttt{AABB::intersect} in the file \texttt{src/accel.cpp}.

\subsection{BVH Construction}

Implement BVH Construction.

\subsection{Direct Illumination Integrator}

Implement the IntersectionTestIntegrator class.

\subsection{Integrate with Refractive Materials}

Add support for refractive materials in the IntersectionTestIntegrator class.

\section{Implementation Details}

\subsection{Ray-Triangle Intersection}

Through the calculation of the intersection point of the ray and the triangle plane in the triangle coordinate system, determine whether the point is inside the triangle.

\subsection{Ray-AABB Intersection}

Calculate the intersection time of the ray and the three pairs of parallel planes of the AABB.
If the ray intersects the AABB, return true.
Otherwise, return false.

\subsection{BVH Construction}

Build the BVH tree using the median split method.

\subsection{Direct Illumination Integrator}

For every pixel it spawns one camera ray per sample; the ray is traced until it hits either an ideal-diffuse or a perfect-refraction surface.  
Specular refractions are followed recursively by sampling the BSDF and spawning a new ray, while diffuse surfaces terminate the path.  
Once a diffuse hit is found, direct illumination from the single point light is computed: a shadow ray is cast toward the light position and, if unoccluded, the reflected radiance is estimated with a simple Lambertian model—albedo from \texttt{bsdf->evaluate} multiplied by the clamped cosine between the surface normal and the light direction.  
All other surface types are ignored, so only direct lighting on ideal-diffuse surfaces contributes to the final pixel color.

\subsection{Integrate with Refractive Materials}

Judge whether it is a total reflection to call refraction or reflection.

\section{Results}

\subsection{Ray-Triangle Intersection \& Ray-AABB Intersection}

\includegraphics[width=0.8\textwidth]{1.png}

\subsection{BVH Intersection}

\includegraphics[width=0.8\textwidth]{2.png}

\subsection{Final}

\includegraphics[width=0.8\textwidth]{3.jpg}

\end{document}
